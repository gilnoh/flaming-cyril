%
% File naaclhlt2015.tex
%

\documentclass[11pt,letterpaper]{article}
\usepackage{naaclhlt2015}
\usepackage{times}
\usepackage{latexsym}
\setlength\titlebox{6.5cm}    % Expanding the titlebox

\title{A extendable, multilingual textual entailment engine by multi-level alignments}

\author{Author 1\\
	    XYZ Company\\
	    111 Anywhere Street\\
	    Mytown, NY 10000, USA\\
	    {\tt author1@xyz.org}
	  \And
	Author 2\\
  	ABC University\\
  	900 Main Street\\
  	Ourcity, PQ, Canada A1A 1T2\\
  {\tt author2@abc.ca}}

\date{}

\begin{document}
\maketitle
\begin{abstract}
 abstract text 
\end{abstract}

\section{Introduction (0.5 page)}
Intro: very short (1/2 page) \\
- TE, state of the art, EOP, remaining problems \\ 

\section{Multi-level alignment (1.5page)}
  %%  MOTIVATION:
  %% * Alignments as "universal (un-)relatedness indicators"
  %% * "Firewall" between relatedness processing and TE decision
  %%   - extensibility
  %%   - multilinguality
  %% * STRUCTURE OF EDA:
  %%  - aligners + feature computers + standard supervised learning
 
\section{Implementation and Evaluation (1.5 page)}
  %% * THIS IS A PILOT STUDY TO TEST THE POTENTIAL OF THE ARCHITECTURE.
  %%  - small set of aligners. (those available for all languages)
  %%  - small set of features. (those applicable to all langauges)
  %%  - evaluation on one dataset (RTE exists for all languages)

\section{Conclusion (0.5 page)} 
   %% * Results: it works! May not be as good as best alignment-based systems
   %%   BUT is extensible, robust, and multilingual
   %% * AND it is available and anyone can use it, not some research prototype (!!!

   Cite.. \cite{Katz:1987}.

%% \section*{Acknowledgments}

%% Do not number the acknowledgment section.
\bibliographystyle{naaclhlt2015}
\bibliography{sem2015_short}

\end{document}
